\PassOptionsToPackage{x11names}{xcolor}
\documentclass[10pt]{beamer}

\usepackage{fontspec}
\setmainfont{Ubuntu}[]
\setsansfont{Ubuntu}[]
\setmonofont{Ubuntu Mono}[]

\usepackage{graphicx}
\graphicspath{ {../img/} }

\usepackage[absolute,overlay]{textpos} % [showboxes]

\usepackage{listings}
\lstdefinelanguage{elixir}{
    morekeywords={case,catch,def,do,else,false,%
        use,alias,receive,timeout,defmacro,defp,%
        for,if,import,defmodule,defprotocol,%
        nil,defmacrop,defoverridable,defimpl,%
        super,fn,raise,true,try,end,with,%
        unless},
    otherkeywords={<-,->, |>, \%\{, \}, \{, \, (, )},
    sensitive=true,
    morecomment=[l]{\#},
    morecomment=[n]{/*}{*/},
	basicstyle=\ttfamily,
	breaklines,
	showstringspaces=false,
	frame=trbl
}

%% https://latex-tutorial.com/color-latex/
\lstset{
  language=elixir,
  keywordstyle=\color{SteelBlue4},
  identifierstyle=\color{DeepSkyBlue3},
  backgroundcolor=\color{Ivory1}
}

\beamertemplatenavigationsymbolsempty

\title{Алгебраические типы данных}

\begin{document}

\begin{frame}
  \frametitle{Алгебраические типы данных}
  ADT (Algebraic data type)
  \par \bigskip
  Определённые операции над типами данных
  \par \bigskip
  позволяют создавать новые типы.
\end{frame}

\begin{frame}
  \frametitle{Алгебраические типы данных}
  Эти операции определены математически:
  \par \bigskip
  \begin{itemize}
  \item произведение (умножение, product)
  \item сумма (перечисление, sum)
  \end{itemize}
\end{frame}

\begin{frame}[fragile]
  \frametitle{Произведение}
  Самое простое произведение -- это кортеж:
  \par \bigskip
  \begin{lstlisting}
  {
    :event,
    title,
    place,
    time,
    participants,
    agenda
  }
  \end{lstlisting}
\end{frame}

\begin{frame}[fragile]
  \frametitle{Произведение}
  В языке OCaml тип кортежа так и обозначается -- через знак умножения:
  \par \bigskip
  \begin{lstlisting}
  String * Place * DateTime * Participant list * Topic list
  \end{lstlisting}
\end{frame}

\begin{frame}[fragile]
  \frametitle{Произведение}
  Структура -- это тоже произведение:
  \par \bigskip
  \begin{lstlisting}
  event = %TS.Event{
    title: "Team Meeting",
    place: place,
    time: time,
    participants: participants,
    agenda: agenda
  }
  \end{lstlisting}
\end{frame}

\begin{frame}[fragile]
  \frametitle{Сумма}
  Теперь посмотрим на \textbf{Topic.priority}
  \par \bigskip
  \begin{lstlisting}
  defmodule Topic do
    @type t() :: %__MODULE__{
            subject: String.t(),
            priority: :high | :medium | :low,
            ...
  \end{lstlisting}
  \par \bigskip
  Здесь есть три вида приоритетов:
  \begin{itemize}
  \item :high
  \item :medium
  \item :low
  \end{itemize}
\end{frame}

\begin{frame}[fragile]
  \frametitle{Сумма}
  Во многих языках для этого есть тип Enum,
  \par \bigskip
  который определяет конечное множество значений.
  \par \bigskip
  В Эликсир такой тип можно описать для Dialyzer:
  \par \bigskip
  \begin{lstlisting}
  @type priority :: :high | :medium | :low
  \end{lstlisting}
\end{frame}

\begin{frame}
  \frametitle{Произведение и сумма}
  Произведение и сумма (product \& sum)
  \par \bigskip
  позволяют описать любые объекты.
\end{frame}

\begin{frame}
  \frametitle{ADT}
  Это математический аппарат,
  \par \bigskip
  который используется при создании системы типов
  \par \bigskip
  в некоторых языках программирования,
  \par \bigskip
  особенно в функциональных языках.
\end{frame}

\begin{frame}
  \frametitle{ADT}
  Языки с полноценной реализацией ADT:
  \par \bigskip
  \begin{itemize}
  \item Haskell
  \item OCaml
  \item Scala
  \item F\#
  \end{itemize}
  \par \bigskip
  имеют некоторые преимущества.
\end{frame}

\begin{frame}[fragile]
  \frametitle{ADT}
  Например, при применении pattern matching к типу Enum
  \par \bigskip
  проверяется, что в шаблонах использованы все возможные значения:
  \par \bigskip
  \begin{lstlisting}
  case topic.priority do
    :high -> do_something
    :medium -> do_something_else
    :low -> do_something_else
  end
  \end{lstlisting}
\end{frame}

\begin{frame}[fragile]
  \frametitle{ADT}
  И если какого-то значения не хватает:
  \par \bigskip
  \begin{lstlisting}
  case topic.priority do
    :high -> do_something
    :medium -> do_something_else
  end
  \end{lstlisting}
  \par \bigskip
  То компилятор выдаёт ошибку "не обработано значение :low".
\end{frame}

\begin{frame}
  \frametitle{ADT}
  Это полезно, когда мы в уже существующий тип
  \par \bigskip
  добавляем новое значение
  \par \bigskip
  и хотим, чтобы оно везде было обработано.
\end{frame}

\begin{frame}
  \frametitle{ADT}
  Увы, компилятор Эликсир и Dialyzer этого не делают.
\end{frame}

\end{document}
