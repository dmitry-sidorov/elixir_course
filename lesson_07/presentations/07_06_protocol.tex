\PassOptionsToPackage{x11names}{xcolor}
\documentclass[10pt]{beamer}

\usepackage{fontspec}
\setmainfont{Ubuntu}[]
\setsansfont{Ubuntu}[]
\setmonofont{Ubuntu Mono}[]

\usepackage{graphicx}
\graphicspath{ {../img/} }

\usepackage[absolute,overlay]{textpos} % [showboxes]

\usepackage{listings}
\lstdefinelanguage{elixir}{
    morekeywords={case,catch,def,do,else,false,%
        use,alias,receive,timeout,defmacro,defp,%
        for,if,import,defmodule,defprotocol,%
        nil,defmacrop,defoverridable,defimpl,%
        super,fn,raise,true,try,end,with,%
        unless},
    otherkeywords={<-,->, |>, \%\{, \}, \{, \, (, )},
    sensitive=true,
    morecomment=[l]{\#},
    morecomment=[n]{/*}{*/},
	basicstyle=\ttfamily,
	breaklines,
	showstringspaces=false,
	frame=trbl
}

%% https://latex-tutorial.com/color-latex/
\lstset{
  language=elixir,
  keywordstyle=\color{SteelBlue4},
  identifierstyle=\color{DeepSkyBlue3},
  backgroundcolor=\color{Ivory1}
}

\beamertemplatenavigationsymbolsempty

\title{Стандартные протоколы}

\begin{document}

\begin{frame}
  \frametitle{Стандартные протоколы}
  В Elixir есть 5 стандартных протоколов:
  \par \bigskip
  \begin{itemize}
  \item Enumerable
  \item Collectable
  \item Inspect
  \item List.Chars
  \item String.Chars
  \end{itemize}
\end{frame}

\begin{frame}
  \frametitle{Enumerable}
  Все коллекции реализуют протокол Enumerable.
  \par \bigskip
  Поэтому функции модулей \textbf{Enum} и \textbf{Stream} работают с:
  \par \bigskip
  \begin{itemize}
  \item List
  \item Keyword List
  \item Map
  \item MapSet
  \item Range
  \item String
  \end{itemize}
\end{frame}

\begin{frame}
  \frametitle{Enumerable}
  Протокол содержит 4 функции:
  \par \bigskip
  \begin{itemize}
  \item count/1
  \item member?/2
  \item reduce/3
  \item slice/1
  \end{itemize}
\end{frame}

\begin{frame}
  \frametitle{Enumerable}
  Напомню, что функций в модуле \textbf{Enum} много:
  \par \bigskip
  \begin{itemize}
  \item map, filter, reduce
  \item all?, any?
  \item sort, find
  \item drop\_while, dedup\_by
  \item chunk\_by, chunk\_while
  \item и другие
  \end{itemize}
  \par \bigskip
  Все они реализуются через 4 функции протокола.
\end{frame}

\begin{frame}
  \frametitle{Enumerable}
  Теоретически достаточно только \textbf{reduce}.
  \par \bigskip
  Все остальное можно реализовать через него.
  \par \bigskip
  Но на практике это не эффективно.
\end{frame}

\begin{frame}
  \frametitle{Enumerable}
  \textbf{Enumerable.reduce} это не то же самое, что \textbf{Enum.reduce}.
  \par \bigskip
  Там более сложная реализация, где можно управлять итерацией:
  останавливать, возобновлять и прерывать.
\end{frame}

\begin{frame}
  \frametitle{Collectable}
  Протокол \textbf{Collectable} описывает,
  \par \bigskip
  как добавить элемент в коллекцию,
  \par \bigskip
  что позволяет преобразовать одну коллекцию в другую.
\end{frame}

\begin{frame}
  \frametitle{Collectable}
  Функции модуля \textbf{Enum} всегда возвращают список.
  \par \bigskip
  Если нам на выходе нужен не список, а другая коллекция,
  \par \bigskip
  то можно использовать \textbf{Enum.into/1},
  \par \bigskip
  который использует протокол Collectable.
\end{frame}

\begin{frame}[fragile]
  \frametitle{Collectable}
  Не редкий случай, когда у нас на входе словарь,
  \par \bigskip
  и на выходе мы хотим получить словарь:
  \par \bigskip
  \begin{lstlisting}
iex(1)> my_map = %{a: 1, b: 2}
%{a: 1, b: 2}
iex(2)> Enum.map(my_map,
...(2)> fn {k, v} -> {k, v + 1} end)
[a: 2, b: 3]
iex(3)> Enum.map(my_map,
...(3)> fn {k, v} -> {k, v + 1} end) |>
...(3)> Enum.into(%{})
%{a: 2, b: 3}
  \end{lstlisting}
\end{frame}

\begin{frame}[fragile]
  \frametitle{Collectable}
  Аналогично это работает с конструкторами списков:
  \par \bigskip
  \begin{lstlisting}
iex(4)> for {k, v} <- my_map, do: {k, v + 1}
[a: 2, b: 3]
iex(5)> for {k, v} <- my_map, into: %{}, do: {k, v + 1}
%{a: 2, b: 3}
  \end{lstlisting}
\end{frame}

\begin{frame}[fragile]
  \frametitle{Inspect}
  Мы не раз видели в iex консоли отображание разных значений.
  \par \bigskip
  В том числе сложных:
  \par \bigskip
  \begin{lstlisting}
iex(1)> MyCalendar.sample_event_typed_struct()
%MyCalendar.Model.TypedStruct.Event{
  title: "Team Meeting",
  ...
  \end{lstlisting}
\end{frame}

\begin{frame}
  \frametitle{Inspect}
  Это делает функция \textbf{Kernel.inspect/2}.
  \par \bigskip
  И это возможно потому, что все типы данных в Эликсир
  \par \bigskip
  реализуют протокол \textbf{Inspect}.
  \par \bigskip
  Каждый тип сам описывает, как он должен быть представлен.
\end{frame}

\begin{frame}[fragile]
  \frametitle{Inspect}
  \textbf{Kernel.inspect} имеет настройки:
  \par \bigskip
  \begin{lstlisting}
inspect(event)
inspect(event, pretty: true)
inspect(event, pretty: true, limit: 3)
inspect(event, pretty: true, width: 10)
  \end{lstlisting}
\end{frame}

\begin{frame}[fragile]
  \frametitle{Inspect}
  И часто применяется при формировании логов:
  \par \bigskip
  \begin{lstlisting}
require Logger
Logger.info("my event is #{inspect(event)}")
Logger.info("my event is #{
  inspect(event, pretty: true, width: 10)}")
  \end{lstlisting}
\end{frame}

\begin{frame}
  \frametitle{Inspect}
  Протокол \textbf{Inspect} по-умолчанию работает для любых данных.
  \par \bigskip
  Но его можно явно переопределить для своих данных.
\end{frame}

\begin{frame}
  Например, если мы не хотим, чтобы приватные данные:
  \par \bigskip
  пароли, ключи, сертификаты,
  \par \bigskip
  отображались в консоли и попадали в логи.
\end{frame}

\begin{frame}
  \frametitle{Title}
  \textbf{bold text} text
  \par \bigskip
  text
\end{frame}

\begin{frame}
  \frametitle{Title}
  \textbf{bold text} text
  \par \bigskip
  text
\end{frame}

\begin{frame}
  \frametitle{Title}
  \textbf{bold text} text
  \par \bigskip
  text
\end{frame}

\begin{frame}
  \frametitle{Title}
  \textbf{bold text} text
  \par \bigskip
  text
\end{frame}

\begin{frame}
  \frametitle{Title}
  \textbf{bold text} text
  \par \bigskip
  text
\end{frame}

\end{document}
