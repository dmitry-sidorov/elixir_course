\PassOptionsToPackage{x11names}{xcolor}
\documentclass[10pt]{beamer}

\usepackage{fontspec}
\setmainfont{Ubuntu}[]
\setsansfont{Ubuntu}[]
\setmonofont{Ubuntu Mono}[]

\usepackage{graphicx}
\graphicspath{ {../img/} }

\usepackage[absolute,overlay]{textpos} % [showboxes]

\usepackage{listings}
\lstdefinelanguage{elixir}{
    morekeywords={case,catch,def,do,else,false,%
        use,alias,receive,timeout,defmacro,defp,%
        for,if,import,defmodule,defprotocol,%
        nil,defmacrop,defoverridable,defimpl,%
        super,fn,raise,true,try,end,with,%
        unless},
    otherkeywords={<-,->, |>, \%\{, \}, \{, \, (, )},
    sensitive=true,
    morecomment=[l]{\#},
    morecomment=[n]{/*}{*/},
	basicstyle=\ttfamily,
	breaklines,
	showstringspaces=false,
	frame=trbl
}

%% https://latex-tutorial.com/color-latex/
\lstset{
  language=elixir,
  keywordstyle=\color{SteelBlue4},
  identifierstyle=\color{DeepSkyBlue3},
  backgroundcolor=\color{Ivory1}
}

\beamertemplatenavigationsymbolsempty

\title{Что такое функциональное программирование?}

\begin{document}

\begin{frame}
  \frametitle{Что такое функциональное программирование?}
  Не так просто ответить на этот вопрос :)
  \par \bigskip
  Есть языки программирования,
  \par \bigskip
  которые традиционно относят к ФП:
  \par \bigskip
  \begin{itemize}
  \item Lisp
  \item Haskell
  \item Standard ML
  \item OCaml
  \item Scala
  \end{itemize}
\end{frame}

\begin{frame}
  \frametitle{Что такое функциональное программирование?}
  Однако многие из них мультипарадигменные.
  \par \bigskip
  То есть, они позволяют писать код не только в стиле ФП.
\end{frame}

\begin{frame}
  \frametitle{Что такое функциональное программирование?}
  С другой стороны есть языки,
  \par \bigskip
  которые изначально не поддерживали ФП,
  \par \bigskip
  но со временем позаимствовали оттуда много элементов:
  \par \bigskip
  \begin{itemize}
  \item Java
  \item C++
  \item Python
  \item JavaScript
  \end{itemize}
\end{frame}

\begin{frame}
  \frametitle{Что такое функциональное программирование?}
  Парадигмы смешиваются в рамках одного языка.
  \par \bigskip
  Но мы всё-таки можем отличить,
  \par \bigskip
  где код написан в стиле ФП, а где нет.
\end{frame}

\begin{frame}
  \frametitle{Элементы ФП}
  Для функционального программирования характерно:
  \par \bigskip
  \begin{itemize}
  \item Иммутабельные данные (Immutability);
  \item Рекурсия, как основной способ итерации по коллекциям;
  \item Функции высшего порядка (HOF);
  \item Анонимные функции (замыкания, лямбды);
  \item Алгебраические типы данных (ADT);
  \item Сопоставление с образцом (Pattern Matching);
  \item Ленивые вычисления (Lazy Evaluation).
  \end{itemize}
  \par \bigskip
  Всё это есть в Эликсир.
\end{frame}

\begin{frame}
  \frametitle{Элементы ФП}
  А вот этого нет в Эликсир:
  \par \bigskip
  \begin{itemize}
  \item Автоматический вывод типов (Type Inference);
  \item Чистые функции и контроль побочных эффектов.
  \end{itemize}
\end{frame}

\end{document}
